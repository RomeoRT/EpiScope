%% Generated by Sphinx.
\def\sphinxdocclass{report}
\documentclass[letterpaper,10pt,english]{sphinxmanual}
\ifdefined\pdfpxdimen
   \let\sphinxpxdimen\pdfpxdimen\else\newdimen\sphinxpxdimen
\fi \sphinxpxdimen=.75bp\relax
\ifdefined\pdfimageresolution
    \pdfimageresolution= \numexpr \dimexpr1in\relax/\sphinxpxdimen\relax
\fi
%% let collapsible pdf bookmarks panel have high depth per default
\PassOptionsToPackage{bookmarksdepth=5}{hyperref}

\PassOptionsToPackage{booktabs}{sphinx}
\PassOptionsToPackage{colorrows}{sphinx}

\PassOptionsToPackage{warn}{textcomp}
\usepackage[utf8]{inputenc}
\ifdefined\DeclareUnicodeCharacter
% support both utf8 and utf8x syntaxes
  \ifdefined\DeclareUnicodeCharacterAsOptional
    \def\sphinxDUC#1{\DeclareUnicodeCharacter{"#1}}
  \else
    \let\sphinxDUC\DeclareUnicodeCharacter
  \fi
  \sphinxDUC{00A0}{\nobreakspace}
  \sphinxDUC{2500}{\sphinxunichar{2500}}
  \sphinxDUC{2502}{\sphinxunichar{2502}}
  \sphinxDUC{2514}{\sphinxunichar{2514}}
  \sphinxDUC{251C}{\sphinxunichar{251C}}
  \sphinxDUC{2572}{\textbackslash}
\fi
\usepackage{cmap}
\usepackage[T1]{fontenc}
\usepackage{amsmath,amssymb,amstext}
\usepackage{babel}



\usepackage{tgtermes}
\usepackage{tgheros}
\renewcommand{\ttdefault}{txtt}



\usepackage[Bjarne]{fncychap}
\usepackage{sphinx}

\fvset{fontsize=auto}
\usepackage{geometry}


% Include hyperref last.
\usepackage{hyperref}
% Fix anchor placement for figures with captions.
\usepackage{hypcap}% it must be loaded after hyperref.
% Set up styles of URL: it should be placed after hyperref.
\urlstyle{same}


\usepackage{sphinxmessages}




\title{Episcope}
\date{Mar 27, 2024}
\release{0.1\sphinxhyphen{}alpha}
\author{Roméo RAMOS\sphinxhyphen{}\sphinxhyphen{}TANGHE, Yosra SAID, Annaelle SARRAZIN, Rachel TECHI, Chloé THIRIET}
\newcommand{\sphinxlogo}{\vbox{}}
\renewcommand{\releasename}{Release}
\makeindex
\begin{document}

\ifdefined\shorthandoff
  \ifnum\catcode`\=\string=\active\shorthandoff{=}\fi
  \ifnum\catcode`\"=\active\shorthandoff{"}\fi
\fi

\pagestyle{empty}
\sphinxmaketitle
\pagestyle{plain}
\sphinxtableofcontents
\pagestyle{normal}
\phantomsection\label{\detokenize{index::doc}}
\noindent{\hspace*{\fill}\sphinxincludegraphics[width=500\sphinxpxdimen,height=110\sphinxpxdimen]{{Logo_final}.png}\hspace*{\fill}}

\sphinxAtStartPar
\sphinxstylestrong{EpiScope} is a GUI featuring ergonomic tools for annotating epileptic seizure videos.
These tools enable practitioners to note directly on their patients’ epileptic seizure videos the various symptoms that appear, thanks to a pre\sphinxhyphen{}configured symptom semiology. This interface generates a .txt text file listing all the symptoms occurring during the seizure in chronological order, as well as a timeline illustrating the patient’s epileptic seizure. The timeline follows a temporal axis (identical to that of the seizure video) and indicates the moment of onset and end of each symptom. Practitioners must also be able to modify the .txt file and the frieze in the event of an oversight or readjustment.

\begin{sphinxadmonition}{note}{Note:}
\sphinxAtStartPar
This project is under active development.
\end{sphinxadmonition}

\sphinxstepscope




\begin{savenotes}\sphinxattablestart
\sphinxthistablewithglobalstyle
\sphinxthistablewithnovlinesstyle
\centering
\begin{tabulary}{\linewidth}[t]{\X{1}{2}\X{1}{2}}
\sphinxtoprule
\sphinxtableatstartofbodyhook\sphinxbottomrule
\end{tabulary}
\sphinxtableafterendhook\par
\sphinxattableend\end{savenotes}

\sphinxstepscope


\chapter{General Interface}
\label{\detokenize{general_interface:module-general_interface_V9}}\label{\detokenize{general_interface:general-interface}}\label{\detokenize{general_interface::doc}}\index{module@\spxentry{module}!general\_interface\_V9@\spxentry{general\_interface\_V9}}\index{general\_interface\_V9@\spxentry{general\_interface\_V9}!module@\spxentry{module}}\begin{description}
\sphinxlineitem{interface générale Episcope contenant :}
\sphinxAtStartPar
lecteur vidéo :
\sphinxhyphen{} lecture video
\sphinxhyphen{} boutons play/pause, skip\textgreater{}\textgreater{}, skip\textless{}\textless{}, revoir
\sphinxhyphen{} avance et recule de 1s et mets en pause
\sphinxhyphen{} son synchronisé
\sphinxhyphen{} bonne vitesse

\sphinxAtStartPar
annotations :
\sphinxhyphen{} pré\sphinxhyphen{}charger des symptomes
\sphinxhyphen{} les menus en cascade a gauche
\sphinxhyphen{} initialisation correcte des symptomes
\sphinxhyphen{} recuperer les symptomes dans une liste
\sphinxhyphen{} recuperer les temps de debut et de fin
\sphinxhyphen{} afficher les symptomes a droite
\sphinxhyphen{} symptomes scrollables
\sphinxhyphen{} pop\sphinxhyphen{}up pour modifier les symptomes

\sphinxAtStartPar
fichiers :
\sphinxhyphen{} générer la frise
\sphinxhyphen{} generer un rapport
\sphinxhyphen{} générer un fichiers de s ymptomes

\end{description}

\sphinxAtStartPar
modification design general et boutons de menus deroulants et de gestion avancement video gestion
tout est en anglais
La document des fonctions est faite
version : 0.9
\index{FriseSymptomes (class in general\_interface\_V9)@\spxentry{FriseSymptomes}\spxextra{class in general\_interface\_V9}}

\begin{fulllineitems}
\phantomsection\label{\detokenize{general_interface:general_interface_V9.FriseSymptomes}}
\pysigstartsignatures
\pysiglinewithargsret{\sphinxbfcode{\sphinxupquote{class\DUrole{w}{ }}}\sphinxcode{\sphinxupquote{general\_interface\_V9.}}\sphinxbfcode{\sphinxupquote{FriseSymptomes}}}{\sphinxparam{\DUrole{n}{InterfaceGenerale}}\sphinxparamcomma \sphinxparam{\DUrole{n}{MenuDeroulant}}}{}
\pysigstopsignatures
\sphinxAtStartPar
Bases: \sphinxcode{\sphinxupquote{object}}

\sphinxAtStartPar
Classe permettant de généré une frise chronologique récapitulant l’ensemble des symptomes présent lors de la crise épileptique
\index{interfaceGenerale (general\_interface\_V9.FriseSymptomes attribute)@\spxentry{interfaceGenerale}\spxextra{general\_interface\_V9.FriseSymptomes attribute}}

\begin{fulllineitems}
\phantomsection\label{\detokenize{general_interface:general_interface_V9.FriseSymptomes.interfaceGenerale}}
\pysigstartsignatures
\pysigline{\sphinxbfcode{\sphinxupquote{interfaceGenerale}}}
\pysigstopsignatures\begin{quote}\begin{description}
\sphinxlineitem{Type}
\sphinxAtStartPar
{\hyperref[\detokenize{general_interface:general_interface_V9.InterfaceGenerale}]{\sphinxcrossref{InterfaceGenerale}}}

\end{description}\end{quote}

\end{fulllineitems}

\index{MenuDeroulant (general\_interface\_V9.FriseSymptomes attribute)@\spxentry{MenuDeroulant}\spxextra{general\_interface\_V9.FriseSymptomes attribute}}

\begin{fulllineitems}
\phantomsection\label{\detokenize{general_interface:general_interface_V9.FriseSymptomes.MenuDeroulant}}
\pysigstartsignatures
\pysigline{\sphinxbfcode{\sphinxupquote{MenuDeroulant}}}
\pysigstopsignatures\begin{quote}\begin{description}
\sphinxlineitem{Type}
\sphinxAtStartPar
{\hyperref[\detokenize{general_interface:general_interface_V9.Menu_symptomes}]{\sphinxcrossref{Menu\_symptomes}}}

\end{description}\end{quote}

\end{fulllineitems}

\index{afficher() (general\_interface\_V9.FriseSymptomes method)@\spxentry{afficher()}\spxextra{general\_interface\_V9.FriseSymptomes method}}

\begin{fulllineitems}
\phantomsection\label{\detokenize{general_interface:general_interface_V9.FriseSymptomes.afficher}}
\pysigstartsignatures
\pysiglinewithargsret{\sphinxbfcode{\sphinxupquote{afficher}}}{}{}
\pysigstopsignatures
\sphinxAtStartPar
Récuperation de la liste de symptomes instanciée dans class interface generale fonction update right panel

\end{fulllineitems}


\end{fulllineitems}

\index{InterfaceGenerale (class in general\_interface\_V9)@\spxentry{InterfaceGenerale}\spxextra{class in general\_interface\_V9}}

\begin{fulllineitems}
\phantomsection\label{\detokenize{general_interface:general_interface_V9.InterfaceGenerale}}
\pysigstartsignatures
\pysiglinewithargsret{\sphinxbfcode{\sphinxupquote{class\DUrole{w}{ }}}\sphinxcode{\sphinxupquote{general\_interface\_V9.}}\sphinxbfcode{\sphinxupquote{InterfaceGenerale}}}{\sphinxparam{\DUrole{n}{fenetre}}}{}
\pysigstopsignatures
\sphinxAtStartPar
Bases: \sphinxcode{\sphinxupquote{object}}

\sphinxAtStartPar
Classe interface Générale qui donne le visuel global de l’interface graphique.

\sphinxAtStartPar
Elle est représenter sous forme de fenetre et permet aux autres classes de s’intégrer dedans.
Elle appel la classe Lecteur\_video, FriseSymptomes et Menu\_symptomes
\begin{description}
\sphinxlineitem{Atributes:}
\sphinxAtStartPar
ListeSymptomes (list): liste des symptomes s’actualisant au fur et a mesure

\sphinxAtStartPar
\sphinxstylestrong{A completer !!!}

\end{description}
\index{get\_current\_video\_time() (general\_interface\_V9.InterfaceGenerale method)@\spxentry{get\_current\_video\_time()}\spxextra{general\_interface\_V9.InterfaceGenerale method}}

\begin{fulllineitems}
\phantomsection\label{\detokenize{general_interface:general_interface_V9.InterfaceGenerale.get_current_video_time}}
\pysigstartsignatures
\pysiglinewithargsret{\sphinxbfcode{\sphinxupquote{get\_current\_video\_time}}}{}{}
\pysigstopsignatures
\sphinxAtStartPar
Permet d’avoir le temps actuel de la vidéo

\end{fulllineitems}

\index{get\_video\_duration() (general\_interface\_V9.InterfaceGenerale method)@\spxentry{get\_video\_duration()}\spxextra{general\_interface\_V9.InterfaceGenerale method}}

\begin{fulllineitems}
\phantomsection\label{\detokenize{general_interface:general_interface_V9.InterfaceGenerale.get_video_duration}}
\pysigstartsignatures
\pysiglinewithargsret{\sphinxbfcode{\sphinxupquote{get\_video\_duration}}}{}{}
\pysigstopsignatures
\sphinxAtStartPar
Permet d’avoir le temps total de la vidéo

\end{fulllineitems}

\index{lire\_fichier() (general\_interface\_V9.InterfaceGenerale method)@\spxentry{lire\_fichier()}\spxextra{general\_interface\_V9.InterfaceGenerale method}}

\begin{fulllineitems}
\phantomsection\label{\detokenize{general_interface:general_interface_V9.InterfaceGenerale.lire_fichier}}
\pysigstartsignatures
\pysiglinewithargsret{\sphinxbfcode{\sphinxupquote{lire\_fichier}}}{\sphinxparam{\DUrole{n}{nom\_fichier}}}{}
\pysigstopsignatures
\sphinxAtStartPar
Ouvre un fichier texte le lit et sauvegarde les informations dans une liste

\end{fulllineitems}

\index{load\_symptoms() (general\_interface\_V9.InterfaceGenerale method)@\spxentry{load\_symptoms()}\spxextra{general\_interface\_V9.InterfaceGenerale method}}

\begin{fulllineitems}
\phantomsection\label{\detokenize{general_interface:general_interface_V9.InterfaceGenerale.load_symptoms}}
\pysigstartsignatures
\pysiglinewithargsret{\sphinxbfcode{\sphinxupquote{load\_symptoms}}}{}{}
\pysigstopsignatures
\sphinxAtStartPar
charge une liste de symptomes a partir d’un fichier

\end{fulllineitems}

\index{ouvrir\_video() (general\_interface\_V9.InterfaceGenerale method)@\spxentry{ouvrir\_video()}\spxextra{general\_interface\_V9.InterfaceGenerale method}}

\begin{fulllineitems}
\phantomsection\label{\detokenize{general_interface:general_interface_V9.InterfaceGenerale.ouvrir_video}}
\pysigstartsignatures
\pysiglinewithargsret{\sphinxbfcode{\sphinxupquote{ouvrir\_video}}}{}{}
\pysigstopsignatures
\sphinxAtStartPar
Ouvre la vidéo en appelant la class Lecteur\_video

\end{fulllineitems}

\index{ouvrir\_video\_noire() (general\_interface\_V9.InterfaceGenerale method)@\spxentry{ouvrir\_video\_noire()}\spxextra{general\_interface\_V9.InterfaceGenerale method}}

\begin{fulllineitems}
\phantomsection\label{\detokenize{general_interface:general_interface_V9.InterfaceGenerale.ouvrir_video_noire}}
\pysigstartsignatures
\pysiglinewithargsret{\sphinxbfcode{\sphinxupquote{ouvrir\_video\_noire}}}{}{}
\pysigstopsignatures
\sphinxAtStartPar
Ouvre la vidéo noire en appelant la class Lecteur\_video

\end{fulllineitems}

\index{rapport() (general\_interface\_V9.InterfaceGenerale method)@\spxentry{rapport()}\spxextra{general\_interface\_V9.InterfaceGenerale method}}

\begin{fulllineitems}
\phantomsection\label{\detokenize{general_interface:general_interface_V9.InterfaceGenerale.rapport}}
\pysigstartsignatures
\pysiglinewithargsret{\sphinxbfcode{\sphinxupquote{rapport}}}{}{}
\pysigstopsignatures
\sphinxAtStartPar
ecrit le rapport de la crise

\end{fulllineitems}

\index{sauvegarde() (general\_interface\_V9.InterfaceGenerale method)@\spxentry{sauvegarde()}\spxextra{general\_interface\_V9.InterfaceGenerale method}}

\begin{fulllineitems}
\phantomsection\label{\detokenize{general_interface:general_interface_V9.InterfaceGenerale.sauvegarde}}
\pysigstartsignatures
\pysiglinewithargsret{\sphinxbfcode{\sphinxupquote{sauvegarde}}}{}{}
\pysigstopsignatures
\sphinxAtStartPar
sauvegarde de la liste de symtptomes dans un fichier texte
fait appel a la classe save

\end{fulllineitems}

\index{supprimer() (general\_interface\_V9.InterfaceGenerale method)@\spxentry{supprimer()}\spxextra{general\_interface\_V9.InterfaceGenerale method}}

\begin{fulllineitems}
\phantomsection\label{\detokenize{general_interface:general_interface_V9.InterfaceGenerale.supprimer}}
\pysigstartsignatures
\pysiglinewithargsret{\sphinxbfcode{\sphinxupquote{supprimer}}}{\sphinxparam{\DUrole{n}{selection}}\sphinxparamcomma \sphinxparam{\DUrole{n}{container}}}{}
\pysigstopsignatures
\end{fulllineitems}

\index{text\_sympt() (general\_interface\_V9.InterfaceGenerale method)@\spxentry{text\_sympt()}\spxextra{general\_interface\_V9.InterfaceGenerale method}}

\begin{fulllineitems}
\phantomsection\label{\detokenize{general_interface:general_interface_V9.InterfaceGenerale.text_sympt}}
\pysigstartsignatures
\pysiglinewithargsret{\sphinxbfcode{\sphinxupquote{text\_sympt}}}{\sphinxparam{\DUrole{n}{sympt}}}{}
\pysigstopsignatures
\sphinxAtStartPar
crée le texta a afficher dans le right\_pannel
\begin{quote}\begin{description}
\sphinxlineitem{Parameters}
\sphinxAtStartPar
\sphinxstyleliteralstrong{\sphinxupquote{sympt}} ({\hyperref[\detokenize{annotation:annotation.class_symptome.Symptome}]{\sphinxcrossref{\sphinxstyleliteralemphasis{\sphinxupquote{Symptome}}}}}) \textendash{} symptome dont on crée le texte

\sphinxlineitem{Returns}
\sphinxAtStartPar
texte a afficher contenant les bonnes infos

\sphinxlineitem{Return type}
\sphinxAtStartPar
text (string)

\end{description}\end{quote}

\end{fulllineitems}

\index{update\_right\_panel() (general\_interface\_V9.InterfaceGenerale method)@\spxentry{update\_right\_panel()}\spxextra{general\_interface\_V9.InterfaceGenerale method}}

\begin{fulllineitems}
\phantomsection\label{\detokenize{general_interface:general_interface_V9.InterfaceGenerale.update_right_panel}}
\pysigstartsignatures
\pysiglinewithargsret{\sphinxbfcode{\sphinxupquote{update\_right\_panel}}}{\sphinxparam{\DUrole{n}{attributs}\DUrole{o}{=}\DUrole{default_value}{{[}{]}}}\sphinxparamcomma \sphinxparam{\DUrole{n}{is\_start\_time}\DUrole{o}{=}\DUrole{default_value}{False}}}{}
\pysigstopsignatures
\sphinxAtStartPar
Permet de gerer l’affichage dans la partie de droite du temps de début/fin des symptomes et gestion du pop\sphinxhyphen{}up pour modifier un symptome.
\begin{quote}\begin{description}
\sphinxlineitem{Parameters}
\sphinxAtStartPar
\sphinxstyleliteralstrong{\sphinxupquote{attributs}} (\sphinxstyleliteralemphasis{\sphinxupquote{list}}) \textendash{} liste d’initialisation du symptome | defaut = {[}{]}

\end{description}\end{quote}

\end{fulllineitems}


\end{fulllineitems}

\index{LecteurVideo (class in general\_interface\_V9)@\spxentry{LecteurVideo}\spxextra{class in general\_interface\_V9}}

\begin{fulllineitems}
\phantomsection\label{\detokenize{general_interface:general_interface_V9.LecteurVideo}}
\pysigstartsignatures
\pysiglinewithargsret{\sphinxbfcode{\sphinxupquote{class\DUrole{w}{ }}}\sphinxcode{\sphinxupquote{general\_interface\_V9.}}\sphinxbfcode{\sphinxupquote{LecteurVideo}}}{\sphinxparam{\DUrole{n}{InterfaceGenerale}}}{}
\pysigstopsignatures
\sphinxAtStartPar
Bases: \sphinxcode{\sphinxupquote{object}}

\sphinxAtStartPar
Classe du lecteur video qui gére les differentes fonctionnalités de la video.

\sphinxAtStartPar
fonctionnalités : ouverture de la vidéo, de la vidéo noire,l’affichage de la vidéo,la synchroisation du son et la gestion des boutons
Elle appelle la classe InterfaceGenerale


\begin{fulllineitems}

\pysigstartsignatures
\pysigline{\sphinxbfcode{\sphinxupquote{\textbackslash{}*\textbackslash{}*A~completer**}}}
\pysigstopsignatures
\end{fulllineitems}

\index{afficher\_menu\_annotations() (general\_interface\_V9.LecteurVideo method)@\spxentry{afficher\_menu\_annotations()}\spxextra{general\_interface\_V9.LecteurVideo method}}

\begin{fulllineitems}
\phantomsection\label{\detokenize{general_interface:general_interface_V9.LecteurVideo.afficher_menu_annotations}}
\pysigstartsignatures
\pysiglinewithargsret{\sphinxbfcode{\sphinxupquote{afficher\_menu\_annotations}}}{\sphinxparam{\DUrole{n}{event}}}{}
\pysigstopsignatures
\sphinxAtStartPar
Récupére les coordonnées d’un clic sur la vidéo. Pour l’intant elle affiche une croix rouge
Mais il ya possibilté d’utiliser cette croix roug pour afficher le symtoms en la survolant
par le curseur

\end{fulllineitems}

\index{afficher\_video() (general\_interface\_V9.LecteurVideo method)@\spxentry{afficher\_video()}\spxextra{general\_interface\_V9.LecteurVideo method}}

\begin{fulllineitems}
\phantomsection\label{\detokenize{general_interface:general_interface_V9.LecteurVideo.afficher_video}}
\pysigstartsignatures
\pysiglinewithargsret{\sphinxbfcode{\sphinxupquote{afficher\_video}}}{}{}
\pysigstopsignatures
\sphinxAtStartPar
permet l’affichage et la lecture de la video:
redimensionnement des frames de la video pour correspondre les dimensions de middle frame
Calcule et affiche le temps totale et le temps écoulé

\end{fulllineitems}

\index{ajouter\_plus\_rouge() (general\_interface\_V9.LecteurVideo method)@\spxentry{ajouter\_plus\_rouge()}\spxextra{general\_interface\_V9.LecteurVideo method}}

\begin{fulllineitems}
\phantomsection\label{\detokenize{general_interface:general_interface_V9.LecteurVideo.ajouter_plus_rouge}}
\pysigstartsignatures
\pysiglinewithargsret{\sphinxbfcode{\sphinxupquote{ajouter\_plus\_rouge}}}{\sphinxparam{\DUrole{n}{canvas}}\sphinxparamcomma \sphinxparam{\DUrole{n}{x}}\sphinxparamcomma \sphinxparam{\DUrole{n}{y}}\sphinxparamcomma \sphinxparam{\DUrole{n}{taille}}}{}
\pysigstopsignatures
\sphinxAtStartPar
Ajoute la croix rouge durat 1 seconde là où on appuie

\end{fulllineitems}

\index{avance\_progress() (general\_interface\_V9.LecteurVideo method)@\spxentry{avance\_progress()}\spxextra{general\_interface\_V9.LecteurVideo method}}

\begin{fulllineitems}
\phantomsection\label{\detokenize{general_interface:general_interface_V9.LecteurVideo.avance_progress}}
\pysigstartsignatures
\pysiglinewithargsret{\sphinxbfcode{\sphinxupquote{avance\_progress}}}{}{}
\pysigstopsignatures
\sphinxAtStartPar
permet d’avancer la video d’une seonde et assure l’avancement du son aussi

\end{fulllineitems}

\index{avancer() (general\_interface\_V9.LecteurVideo method)@\spxentry{avancer()}\spxextra{general\_interface\_V9.LecteurVideo method}}

\begin{fulllineitems}
\phantomsection\label{\detokenize{general_interface:general_interface_V9.LecteurVideo.avancer}}
\pysigstartsignatures
\pysiglinewithargsret{\sphinxbfcode{\sphinxupquote{avancer}}}{}{}
\pysigstopsignatures
\end{fulllineitems}

\index{charger\_son\_video() (general\_interface\_V9.LecteurVideo method)@\spxentry{charger\_son\_video()}\spxextra{general\_interface\_V9.LecteurVideo method}}

\begin{fulllineitems}
\phantomsection\label{\detokenize{general_interface:general_interface_V9.LecteurVideo.charger_son_video}}
\pysigstartsignatures
\pysiglinewithargsret{\sphinxbfcode{\sphinxupquote{charger\_son\_video}}}{\sphinxparam{\DUrole{n}{file\_path}}}{}
\pysigstopsignatures
\sphinxAtStartPar
Arrête le son précédent, et fait l’extraction et le chargement
de la nouvelle piste audio associée à la vidéo sélectionnée
Elle fait aussi le nettoyage des ressources utilisées pendant le processus.

\end{fulllineitems}

\index{configurer\_barre\_progression() (general\_interface\_V9.LecteurVideo method)@\spxentry{configurer\_barre\_progression()}\spxextra{general\_interface\_V9.LecteurVideo method}}

\begin{fulllineitems}
\phantomsection\label{\detokenize{general_interface:general_interface_V9.LecteurVideo.configurer_barre_progression}}
\pysigstartsignatures
\pysiglinewithargsret{\sphinxbfcode{\sphinxupquote{configurer\_barre\_progression}}}{}{}
\pysigstopsignatures
\end{fulllineitems}

\index{format\_duree() (general\_interface\_V9.LecteurVideo method)@\spxentry{format\_duree()}\spxextra{general\_interface\_V9.LecteurVideo method}}

\begin{fulllineitems}
\phantomsection\label{\detokenize{general_interface:general_interface_V9.LecteurVideo.format_duree}}
\pysigstartsignatures
\pysiglinewithargsret{\sphinxbfcode{\sphinxupquote{format\_duree}}}{\sphinxparam{\DUrole{n}{seconds}}}{}
\pysigstopsignatures
\sphinxAtStartPar
convertir une durée en secondes en un format de temps plus lisible,
sous la forme de minutes et secondes.

\end{fulllineitems}

\index{manual\_update\_progress() (general\_interface\_V9.LecteurVideo method)@\spxentry{manual\_update\_progress()}\spxextra{general\_interface\_V9.LecteurVideo method}}

\begin{fulllineitems}
\phantomsection\label{\detokenize{general_interface:general_interface_V9.LecteurVideo.manual_update_progress}}
\pysigstartsignatures
\pysiglinewithargsret{\sphinxbfcode{\sphinxupquote{manual\_update\_progress}}}{\sphinxparam{\DUrole{n}{value}}}{}
\pysigstopsignatures
\end{fulllineitems}

\index{mettre\_a\_jour\_frame\_video() (general\_interface\_V9.LecteurVideo method)@\spxentry{mettre\_a\_jour\_frame\_video()}\spxextra{general\_interface\_V9.LecteurVideo method}}

\begin{fulllineitems}
\phantomsection\label{\detokenize{general_interface:general_interface_V9.LecteurVideo.mettre_a_jour_frame_video}}
\pysigstartsignatures
\pysiglinewithargsret{\sphinxbfcode{\sphinxupquote{mettre\_a\_jour\_frame\_video}}}{\sphinxparam{\DUrole{n}{frame}}}{}
\pysigstopsignatures
\sphinxAtStartPar
permet de s’assurer que chaque frame de la vidéo est correctement traitée,
redimensionnée et affichée dans l’interface utilisateur

\end{fulllineitems}

\index{mettre\_a\_jour\_temps\_video() (general\_interface\_V9.LecteurVideo method)@\spxentry{mettre\_a\_jour\_temps\_video()}\spxextra{general\_interface\_V9.LecteurVideo method}}

\begin{fulllineitems}
\phantomsection\label{\detokenize{general_interface:general_interface_V9.LecteurVideo.mettre_a_jour_temps_video}}
\pysigstartsignatures
\pysiglinewithargsret{\sphinxbfcode{\sphinxupquote{mettre\_a\_jour\_temps\_video}}}{}{}
\pysigstopsignatures
\sphinxAtStartPar
Assure que le temps de la vidéo affichée est régulièrement mis à jour et que la lecture de la vidéo continue
de manière fluide. Elle assure la sychronisation de  l’avancement de la vidéo avec le temps réel

\end{fulllineitems}

\index{ouvrir\_video() (general\_interface\_V9.LecteurVideo method)@\spxentry{ouvrir\_video()}\spxextra{general\_interface\_V9.LecteurVideo method}}

\begin{fulllineitems}
\phantomsection\label{\detokenize{general_interface:general_interface_V9.LecteurVideo.ouvrir_video}}
\pysigstartsignatures
\pysiglinewithargsret{\sphinxbfcode{\sphinxupquote{ouvrir\_video}}}{}{}
\pysigstopsignatures
\sphinxAtStartPar
Ouvre une vidéo depuis l’explorateur du fichier
une fois choisie, elle se met en lecture dans middle\_frame

\end{fulllineitems}

\index{ouvrir\_video\_noire() (general\_interface\_V9.LecteurVideo method)@\spxentry{ouvrir\_video\_noire()}\spxextra{general\_interface\_V9.LecteurVideo method}}

\begin{fulllineitems}
\phantomsection\label{\detokenize{general_interface:general_interface_V9.LecteurVideo.ouvrir_video_noire}}
\pysigstartsignatures
\pysiglinewithargsret{\sphinxbfcode{\sphinxupquote{ouvrir\_video\_noire}}}{}{}
\pysigstopsignatures
\sphinxAtStartPar
Ouvre la vidéo noire directement qui doit étre dans le meme dossier que le code

\end{fulllineitems}

\index{pause\_lecture() (general\_interface\_V9.LecteurVideo method)@\spxentry{pause\_lecture()}\spxextra{general\_interface\_V9.LecteurVideo method}}

\begin{fulllineitems}
\phantomsection\label{\detokenize{general_interface:general_interface_V9.LecteurVideo.pause_lecture}}
\pysigstartsignatures
\pysiglinewithargsret{\sphinxbfcode{\sphinxupquote{pause\_lecture}}}{}{}
\pysigstopsignatures
\sphinxAtStartPar
Cette fonction gère l’état de pause et reprise de la vidéo

\end{fulllineitems}

\index{preparer\_son\_video() (general\_interface\_V9.LecteurVideo method)@\spxentry{preparer\_son\_video()}\spxextra{general\_interface\_V9.LecteurVideo method}}

\begin{fulllineitems}
\phantomsection\label{\detokenize{general_interface:general_interface_V9.LecteurVideo.preparer_son_video}}
\pysigstartsignatures
\pysiglinewithargsret{\sphinxbfcode{\sphinxupquote{preparer\_son\_video}}}{\sphinxparam{\DUrole{n}{file\_path}}}{}
\pysigstopsignatures
\sphinxAtStartPar
Prépare le son de la vidéo à partir de la vidéo originale.

\end{fulllineitems}

\index{recule\_progress() (general\_interface\_V9.LecteurVideo method)@\spxentry{recule\_progress()}\spxextra{general\_interface\_V9.LecteurVideo method}}

\begin{fulllineitems}
\phantomsection\label{\detokenize{general_interface:general_interface_V9.LecteurVideo.recule_progress}}
\pysigstartsignatures
\pysiglinewithargsret{\sphinxbfcode{\sphinxupquote{recule\_progress}}}{}{}
\pysigstopsignatures
\sphinxAtStartPar
Permet de reculer de 1 seconde dans la video en assurant la synchronisation du son avec la video

\end{fulllineitems}

\index{revoir\_video() (general\_interface\_V9.LecteurVideo method)@\spxentry{revoir\_video()}\spxextra{general\_interface\_V9.LecteurVideo method}}

\begin{fulllineitems}
\phantomsection\label{\detokenize{general_interface:general_interface_V9.LecteurVideo.revoir_video}}
\pysigstartsignatures
\pysiglinewithargsret{\sphinxbfcode{\sphinxupquote{revoir\_video}}}{}{}
\pysigstopsignatures
\sphinxAtStartPar
Permet de revoir la video depuis le début

\end{fulllineitems}


\end{fulllineitems}

\index{Menu\_symptomes (class in general\_interface\_V9)@\spxentry{Menu\_symptomes}\spxextra{class in general\_interface\_V9}}

\begin{fulllineitems}
\phantomsection\label{\detokenize{general_interface:general_interface_V9.Menu_symptomes}}
\pysigstartsignatures
\pysiglinewithargsret{\sphinxbfcode{\sphinxupquote{class\DUrole{w}{ }}}\sphinxcode{\sphinxupquote{general\_interface\_V9.}}\sphinxbfcode{\sphinxupquote{Menu\_symptomes}}}{\sphinxparam{\DUrole{n}{master}}\sphinxparamcomma \sphinxparam{\DUrole{n}{interface\_generale}}\sphinxparamcomma \sphinxparam{\DUrole{n}{couleur}}\sphinxparamcomma \sphinxparam{\DUrole{n}{bordure}}\sphinxparamcomma \sphinxparam{\DUrole{n}{largeur}}}{}
\pysigstopsignatures
\sphinxAtStartPar
Bases: \sphinxcode{\sphinxupquote{CTkFrame}}

\sphinxAtStartPar
Classe permettant d’instancier les menus déroulants rassemblant les symptomes dans une frame qui se situe sur la gauche de l’interface
\index{master (general\_interface\_V9.Menu\_symptomes attribute)@\spxentry{master}\spxextra{general\_interface\_V9.Menu\_symptomes attribute}}

\begin{fulllineitems}
\phantomsection\label{\detokenize{general_interface:general_interface_V9.Menu_symptomes.master}}
\pysigstartsignatures
\pysigline{\sphinxbfcode{\sphinxupquote{master}}}
\pysigstopsignatures\begin{quote}\begin{description}
\sphinxlineitem{Type}
\sphinxAtStartPar
any

\end{description}\end{quote}

\end{fulllineitems}



\begin{fulllineitems}

\pysigstartsignatures
\pysigline{\sphinxbfcode{\sphinxupquote{interface~generale}}}
\pysigstopsignatures\begin{quote}\begin{description}
\sphinxlineitem{Type}
\sphinxAtStartPar
{\hyperref[\detokenize{general_interface:general_interface_V9.InterfaceGenerale}]{\sphinxcrossref{InterfaceGenerale}}}

\end{description}\end{quote}

\end{fulllineitems}

\index{couleur (general\_interface\_V9.Menu\_symptomes attribute)@\spxentry{couleur}\spxextra{general\_interface\_V9.Menu\_symptomes attribute}}

\begin{fulllineitems}
\phantomsection\label{\detokenize{general_interface:general_interface_V9.Menu_symptomes.couleur}}
\pysigstartsignatures
\pysigline{\sphinxbfcode{\sphinxupquote{couleur}}}
\pysigstopsignatures
\sphinxAtStartPar
couleur du fond
\begin{quote}\begin{description}
\sphinxlineitem{Type}
\sphinxAtStartPar
str

\end{description}\end{quote}

\end{fulllineitems}

\index{bordure (general\_interface\_V9.Menu\_symptomes attribute)@\spxentry{bordure}\spxextra{general\_interface\_V9.Menu\_symptomes attribute}}

\begin{fulllineitems}
\phantomsection\label{\detokenize{general_interface:general_interface_V9.Menu_symptomes.bordure}}
\pysigstartsignatures
\pysigline{\sphinxbfcode{\sphinxupquote{bordure}}}
\pysigstopsignatures\begin{quote}\begin{description}
\sphinxlineitem{Type}
\sphinxAtStartPar
int

\end{description}\end{quote}

\end{fulllineitems}

\index{largeur (general\_interface\_V9.Menu\_symptomes attribute)@\spxentry{largeur}\spxextra{general\_interface\_V9.Menu\_symptomes attribute}}

\begin{fulllineitems}
\phantomsection\label{\detokenize{general_interface:general_interface_V9.Menu_symptomes.largeur}}
\pysigstartsignatures
\pysigline{\sphinxbfcode{\sphinxupquote{largeur}}}
\pysigstopsignatures\begin{quote}\begin{description}
\sphinxlineitem{Type}
\sphinxAtStartPar
int

\end{description}\end{quote}

\end{fulllineitems}

\index{create\_dropdown\_menus() (general\_interface\_V9.Menu\_symptomes method)@\spxentry{create\_dropdown\_menus()}\spxextra{general\_interface\_V9.Menu\_symptomes method}}

\begin{fulllineitems}
\phantomsection\label{\detokenize{general_interface:general_interface_V9.Menu_symptomes.create_dropdown_menus}}
\pysigstartsignatures
\pysiglinewithargsret{\sphinxbfcode{\sphinxupquote{create\_dropdown\_menus}}}{\sphinxparam{\DUrole{n}{master}}\sphinxparamcomma \sphinxparam{\DUrole{n}{largeur}}}{}
\pysigstopsignatures
\sphinxAtStartPar
Crée un menu déroulants contenant les différents symptomes classés selon s’ils sont objectifs ou subjectifs.
\begin{quote}\begin{description}
\sphinxlineitem{Parameters}
\sphinxAtStartPar
\sphinxstyleliteralstrong{\sphinxupquote{master}} (\sphinxstyleliteralemphasis{\sphinxupquote{fenetre}}) \textendash{} fenetre dans laquelle on veut afficher le sous menu déroulant

\end{description}\end{quote}

\end{fulllineitems}

\index{create\_submenu() (general\_interface\_V9.Menu\_symptomes method)@\spxentry{create\_submenu()}\spxextra{general\_interface\_V9.Menu\_symptomes method}}

\begin{fulllineitems}
\phantomsection\label{\detokenize{general_interface:general_interface_V9.Menu_symptomes.create_submenu}}
\pysigstartsignatures
\pysiglinewithargsret{\sphinxbfcode{\sphinxupquote{create\_submenu}}}{\sphinxparam{\DUrole{n}{parent\_menu}}\sphinxparamcomma \sphinxparam{\DUrole{n}{symptom}}\sphinxparamcomma \sphinxparam{\DUrole{n}{sub\_symptoms}}\sphinxparamcomma \sphinxparam{\DUrole{n}{my\_font}}\sphinxparamcomma \sphinxparam{\DUrole{n}{on\_select}}}{}
\pysigstopsignatures
\sphinxAtStartPar
Crée les sous menus déroulants après avoir sélectionné un symptome.
Les éléments de ce menu precisent la localisation du symptome sur le corps et la latéralité.
\begin{quote}\begin{description}
\sphinxlineitem{Parameters}\begin{itemize}
\item {} 
\sphinxAtStartPar
\sphinxstyleliteralstrong{\sphinxupquote{parent\_menu}} (\sphinxstyleliteralemphasis{\sphinxupquote{Menu}}) \textendash{} Menu déroulant lié au symptome sélectionné

\item {} 
\sphinxAtStartPar
\sphinxstyleliteralstrong{\sphinxupquote{symptom}} (\sphinxstyleliteralemphasis{\sphinxupquote{symptome}}) \textendash{} Symptome sélectionné

\item {} 
\sphinxAtStartPar
\sphinxstyleliteralstrong{\sphinxupquote{sub\_symptoms}} (\sphinxstyleliteralemphasis{\sphinxupquote{list}}) \textendash{} liste complémantaire au symptôme sélectionné (indique la position/latéralisation)

\item {} 
\sphinxAtStartPar
\sphinxstyleliteralstrong{\sphinxupquote{my\_font}} (\sphinxstyleliteralemphasis{\sphinxupquote{Font}}) \textendash{} indique la taille de la police d’écriture

\item {} 
\sphinxAtStartPar
\sphinxstyleliteralstrong{\sphinxupquote{on\_select}} (\sphinxstyleliteralemphasis{\sphinxupquote{function}}) \textendash{} commande de selection des symptomes

\end{itemize}

\end{description}\end{quote}

\end{fulllineitems}

\index{on\_select\_obj() (general\_interface\_V9.Menu\_symptomes method)@\spxentry{on\_select\_obj()}\spxextra{general\_interface\_V9.Menu\_symptomes method}}

\begin{fulllineitems}
\phantomsection\label{\detokenize{general_interface:general_interface_V9.Menu_symptomes.on_select_obj}}
\pysigstartsignatures
\pysiglinewithargsret{\sphinxbfcode{\sphinxupquote{on\_select\_obj}}}{\sphinxparam{\DUrole{n}{selection}}}{}
\pysigstopsignatures
\sphinxAtStartPar
Récupère des données lié à la video pour obtenir le temps actuel
\begin{quote}\begin{description}
\sphinxlineitem{Parameters}
\sphinxAtStartPar
\sphinxstyleliteralstrong{\sphinxupquote{selection}} (\sphinxstyleliteralemphasis{\sphinxupquote{path}}) \textendash{} Symptome sélectionné

\end{description}\end{quote}

\end{fulllineitems}

\index{on\_select\_subj() (general\_interface\_V9.Menu\_symptomes method)@\spxentry{on\_select\_subj()}\spxextra{general\_interface\_V9.Menu\_symptomes method}}

\begin{fulllineitems}
\phantomsection\label{\detokenize{general_interface:general_interface_V9.Menu_symptomes.on_select_subj}}
\pysigstartsignatures
\pysiglinewithargsret{\sphinxbfcode{\sphinxupquote{on\_select\_subj}}}{\sphinxparam{\DUrole{n}{selection}}}{}
\pysigstopsignatures
\sphinxAtStartPar
Récupère des données lié à la video pour obtenir le temps actuel
\begin{quote}\begin{description}
\sphinxlineitem{Parameters}
\sphinxAtStartPar
\sphinxstyleliteralstrong{\sphinxupquote{selection}} (\sphinxstyleliteralemphasis{\sphinxupquote{path}}) \textendash{} Symptome sélectionné

\end{description}\end{quote}

\end{fulllineitems}

\index{read\_symptoms\_from\_file() (general\_interface\_V9.Menu\_symptomes method)@\spxentry{read\_symptoms\_from\_file()}\spxextra{general\_interface\_V9.Menu\_symptomes method}}

\begin{fulllineitems}
\phantomsection\label{\detokenize{general_interface:general_interface_V9.Menu_symptomes.read_symptoms_from_file}}
\pysigstartsignatures
\pysiglinewithargsret{\sphinxbfcode{\sphinxupquote{read\_symptoms\_from\_file}}}{\sphinxparam{\DUrole{n}{file\_name}}}{}
\pysigstopsignatures
\sphinxAtStartPar
Lit les fichiers textes contenant la liste des symptomes pour remplir les menus

\sphinxAtStartPar
Les symptomes doivent etre ecrits avec des séparateurs spécifiques
\sphinxstyleemphasis{example :  Negative myoclonus{[}Oriented(Left;Right;Bilateral);Hand/Superior limb;Foot/Inferior limb{]}}
\begin{quote}\begin{description}
\sphinxlineitem{Parameters}
\sphinxAtStartPar
\sphinxstyleliteralstrong{\sphinxupquote{file\_name}} (\sphinxstyleliteralemphasis{\sphinxupquote{string}}) \textendash{} chemin du fichier

\sphinxlineitem{Returns}
\sphinxAtStartPar
titre du menu
symptoms (list): liste contenant les symptomes pour remplir les menus

\sphinxlineitem{Return type}
\sphinxAtStartPar
title (string)

\end{description}\end{quote}

\end{fulllineitems}


\end{fulllineitems}


\sphinxstepscope


\chapter{frise package}
\label{\detokenize{frise:frise-package}}\label{\detokenize{frise::doc}}

\section{Submodules}
\label{\detokenize{frise:submodules}}

\section{frise.ecriture\_fichier module}
\label{\detokenize{frise:module-frise.ecriture_fichier}}\label{\detokenize{frise:frise-ecriture-fichier-module}}\index{module@\spxentry{module}!frise.ecriture\_fichier@\spxentry{frise.ecriture\_fichier}}\index{frise.ecriture\_fichier@\spxentry{frise.ecriture\_fichier}!module@\spxentry{module}}
\sphinxAtStartPar
Ce module contient des fonctions pour créer et éditer les fichiers ‘.txt’ de sortie avec les symptomes
\index{EcrireListeSymptome() (in module frise.ecriture\_fichier)@\spxentry{EcrireListeSymptome()}\spxextra{in module frise.ecriture\_fichier}}

\begin{fulllineitems}
\phantomsection\label{\detokenize{frise:frise.ecriture_fichier.EcrireListeSymptome}}
\pysigstartsignatures
\pysiglinewithargsret{\sphinxcode{\sphinxupquote{frise.ecriture\_fichier.}}\sphinxbfcode{\sphinxupquote{EcrireListeSymptome}}}{\sphinxparam{\DUrole{n}{listeSymptome}}\sphinxparamcomma \sphinxparam{\DUrole{n}{nomfichier}}}{}
\pysigstopsignatures
\sphinxAtStartPar
Ecrit une liste de symptomes dans un fichier texte

\sphinxAtStartPar
Chaque symptome est écrit sur une ligne
Fait appel à la fonction EcrireSymptome
\begin{quote}\begin{description}
\sphinxlineitem{Parameters}\begin{itemize}
\item {} 
\sphinxAtStartPar
\sphinxstyleliteralstrong{\sphinxupquote{symptome}} (\sphinxcode{\sphinxupquote{list}} of \sphinxcode{\sphinxupquote{str}}) \textendash{} liste des symptomes

\item {} 
\sphinxAtStartPar
\sphinxstyleliteralstrong{\sphinxupquote{nomfichier}} (\sphinxstyleliteralemphasis{\sphinxupquote{str}}) \textendash{} chemin du fichier

\end{itemize}

\sphinxlineitem{Returns}
\sphinxAtStartPar
None

\end{description}\end{quote}

\end{fulllineitems}

\index{EcrireMetaData() (in module frise.ecriture\_fichier)@\spxentry{EcrireMetaData()}\spxextra{in module frise.ecriture\_fichier}}

\begin{fulllineitems}
\phantomsection\label{\detokenize{frise:frise.ecriture_fichier.EcrireMetaData}}
\pysigstartsignatures
\pysiglinewithargsret{\sphinxcode{\sphinxupquote{frise.ecriture\_fichier.}}\sphinxbfcode{\sphinxupquote{EcrireMetaData}}}{\sphinxparam{\DUrole{n}{Meta}}\sphinxparamcomma \sphinxparam{\DUrole{n}{nomfichier}}}{}
\pysigstopsignatures
\sphinxAtStartPar
Ecrit les metadata d’une annotation dans un fichier texte sous la forme : heure réelle :    patient :       praticien :     date d’annotation :
\begin{quote}\begin{description}
\sphinxlineitem{Parameters}\begin{itemize}
\item {} 
\sphinxAtStartPar
\sphinxstyleliteralstrong{\sphinxupquote{(}} (\sphinxstyleliteralemphasis{\sphinxupquote{Meta}}) \textendash{} obj: MetaData): Les métadatas

\item {} 
\sphinxAtStartPar
\sphinxstyleliteralstrong{\sphinxupquote{nomfichier}} (\sphinxstyleliteralemphasis{\sphinxupquote{str}}) \textendash{} chemin du fichier

\end{itemize}

\sphinxlineitem{Returns}
\sphinxAtStartPar
None

\end{description}\end{quote}

\end{fulllineitems}

\index{EcrireSymptome() (in module frise.ecriture\_fichier)@\spxentry{EcrireSymptome()}\spxextra{in module frise.ecriture\_fichier}}

\begin{fulllineitems}
\phantomsection\label{\detokenize{frise:frise.ecriture_fichier.EcrireSymptome}}
\pysigstartsignatures
\pysiglinewithargsret{\sphinxcode{\sphinxupquote{frise.ecriture\_fichier.}}\sphinxbfcode{\sphinxupquote{EcrireSymptome}}}{\sphinxparam{\DUrole{n}{symptome}}\sphinxparamcomma \sphinxparam{\DUrole{n}{nomfichier}}}{}
\pysigstopsignatures
\sphinxAtStartPar
Ecrit un symptome dans un fichier texte dont on specifie le nom
\begin{quote}\begin{description}
\sphinxlineitem{Parameters}\begin{itemize}
\item {} 
\sphinxAtStartPar
\sphinxstyleliteralstrong{\sphinxupquote{symptome}} (\sphinxcode{\sphinxupquote{Symptome}}) \textendash{} symptome à écrire

\item {} 
\sphinxAtStartPar
\sphinxstyleliteralstrong{\sphinxupquote{nomfichier}} (\sphinxstyleliteralemphasis{\sphinxupquote{str}}) \textendash{} chemin du fichier

\end{itemize}

\sphinxlineitem{Returns}
\sphinxAtStartPar
None

\end{description}\end{quote}

\end{fulllineitems}

\index{ecrire\_rapport() (in module frise.ecriture\_fichier)@\spxentry{ecrire\_rapport()}\spxextra{in module frise.ecriture\_fichier}}

\begin{fulllineitems}
\phantomsection\label{\detokenize{frise:frise.ecriture_fichier.ecrire_rapport}}
\pysigstartsignatures
\pysiglinewithargsret{\sphinxcode{\sphinxupquote{frise.ecriture\_fichier.}}\sphinxbfcode{\sphinxupquote{ecrire\_rapport}}}{\sphinxparam{\DUrole{n}{Symptom\_list}}\sphinxparamcomma \sphinxparam{\DUrole{n}{filename}}}{}
\pysigstopsignatures
\sphinxAtStartPar
Ecrit les symptomes dans le rapport lisible par les médecins
\begin{quote}\begin{description}
\sphinxlineitem{Parameters}\begin{itemize}
\item {} 
\sphinxAtStartPar
\sphinxstyleliteralstrong{\sphinxupquote{Symptom\_list}} (\sphinxcode{\sphinxupquote{list}} of \sphinxcode{\sphinxupquote{Symptome}}) \textendash{} liste des symptomes a écrire

\item {} 
\sphinxAtStartPar
\sphinxstyleliteralstrong{\sphinxupquote{filename}} (\sphinxstyleliteralemphasis{\sphinxupquote{str}}) \textendash{} chemein du fichier a ecrire

\end{itemize}

\end{description}\end{quote}

\end{fulllineitems}

\index{format() (in module frise.ecriture\_fichier)@\spxentry{format()}\spxextra{in module frise.ecriture\_fichier}}

\begin{fulllineitems}
\phantomsection\label{\detokenize{frise:frise.ecriture_fichier.format}}
\pysigstartsignatures
\pysiglinewithargsret{\sphinxcode{\sphinxupquote{frise.ecriture\_fichier.}}\sphinxbfcode{\sphinxupquote{format}}}{\sphinxparam{\DUrole{n}{data}}\sphinxparamcomma \sphinxparam{\DUrole{n}{caracteres}}}{}
\pysigstopsignatures
\sphinxAtStartPar
supprime les caracteres speciaux d’une liste de string
\begin{quote}\begin{description}
\sphinxlineitem{Parameters}\begin{itemize}
\item {} 
\sphinxAtStartPar
\sphinxstyleliteralstrong{\sphinxupquote{data}} (\sphinxcode{\sphinxupquote{list}} of \sphinxcode{\sphinxupquote{str}}) \textendash{} liste de strings a traiter

\item {} 
\sphinxAtStartPar
\sphinxstyleliteralstrong{\sphinxupquote{caracteres}} (\sphinxstyleliteralemphasis{\sphinxupquote{List}}) \textendash{} liste des caracteres et de leur remplacement, de la forme {[}(“C1”, “C2”){]}

\end{itemize}

\sphinxlineitem{Returns}
\sphinxAtStartPar
liste de strings traitée

\sphinxlineitem{Return type}
\sphinxAtStartPar
new\_data (\sphinxcode{\sphinxupquote{list}} of \sphinxcode{\sphinxupquote{str}})

\end{description}\end{quote}

\end{fulllineitems}



\section{frise.fonctions\_frise module}
\label{\detokenize{frise:module-frise.fonctions_frise}}\label{\detokenize{frise:frise-fonctions-frise-module}}\index{module@\spxentry{module}!frise.fonctions\_frise@\spxentry{frise.fonctions\_frise}}\index{frise.fonctions\_frise@\spxentry{frise.fonctions\_frise}!module@\spxentry{module}}\index{afficher\_frise() (in module frise.fonctions\_frise)@\spxentry{afficher\_frise()}\spxextra{in module frise.fonctions\_frise}}

\begin{fulllineitems}
\phantomsection\label{\detokenize{frise:frise.fonctions_frise.afficher_frise}}
\pysigstartsignatures
\pysiglinewithargsret{\sphinxcode{\sphinxupquote{frise.fonctions\_frise.}}\sphinxbfcode{\sphinxupquote{afficher\_frise}}}{\sphinxparam{\DUrole{n}{liste}}}{}
\pysigstopsignatures
\sphinxAtStartPar
Affiche la frise chronologique des symptômes.
:param liste (: obj:list of :obj:list): liste des symptomes où chaque élément est une liste {[}Name, début, fin, Lateralization, seg corporel, orientation, attribut suppl, Comment, tdeb\_str, tfin\_str{]}.

\end{fulllineitems}

\index{chercherElt() (in module frise.fonctions\_frise)@\spxentry{chercherElt()}\spxextra{in module frise.fonctions\_frise}}

\begin{fulllineitems}
\phantomsection\label{\detokenize{frise:frise.fonctions_frise.chercherElt}}
\pysigstartsignatures
\pysiglinewithargsret{\sphinxcode{\sphinxupquote{frise.fonctions\_frise.}}\sphinxbfcode{\sphinxupquote{chercherElt}}}{\sphinxparam{\DUrole{n}{list}}}{}
\pysigstopsignatures
\sphinxAtStartPar
Cherche s’il y a un élément manquant dans une liste de chiffre 0,1,2,3,4,5,…

\end{fulllineitems}

\index{chevauchement() (in module frise.fonctions\_frise)@\spxentry{chevauchement()}\spxextra{in module frise.fonctions\_frise}}

\begin{fulllineitems}
\phantomsection\label{\detokenize{frise:frise.fonctions_frise.chevauchement}}
\pysigstartsignatures
\pysiglinewithargsret{\sphinxcode{\sphinxupquote{frise.fonctions\_frise.}}\sphinxbfcode{\sphinxupquote{chevauchement}}}{\sphinxparam{\DUrole{n}{liste}}\sphinxparamcomma \sphinxparam{\DUrole{n}{symp}}\sphinxparamcomma \sphinxparam{\DUrole{n}{current\_index}}\sphinxparamcomma \sphinxparam{\DUrole{n}{levels}}}{}
\pysigstopsignatures
\sphinxAtStartPar
Gère le problème de superposition visuelle des symptômes.
\begin{quote}\begin{description}
\sphinxlineitem{Parameters}\begin{itemize}
\item {} 
\sphinxAtStartPar
\sphinxstyleliteralstrong{\sphinxupquote{liste}} (\sphinxstyleliteralemphasis{\sphinxupquote{list}}) \textendash{} liste des symptomes avec début et fin

\item {} 
\sphinxAtStartPar
\sphinxstyleliteralstrong{\sphinxupquote{symp}} (\sphinxcode{\sphinxupquote{Symptome}}) \textendash{} symptome actuel.

\item {} 
\sphinxAtStartPar
\sphinxstyleliteralstrong{\sphinxupquote{current\_index}} (\sphinxstyleliteralemphasis{\sphinxupquote{int}}) \textendash{} indice du symp actuel dans la liste.

\end{itemize}

\sphinxlineitem{Returns}
\sphinxAtStartPar
niveau y où afficher le rectangle.

\sphinxlineitem{Return type}
\sphinxAtStartPar
int

\end{description}\end{quote}

\end{fulllineitems}

\index{on\_text\_click() (in module frise.fonctions\_frise)@\spxentry{on\_text\_click()}\spxextra{in module frise.fonctions\_frise}}

\begin{fulllineitems}
\phantomsection\label{\detokenize{frise:frise.fonctions_frise.on_text_click}}
\pysigstartsignatures
\pysiglinewithargsret{\sphinxcode{\sphinxupquote{frise.fonctions\_frise.}}\sphinxbfcode{\sphinxupquote{on\_text\_click}}}{\sphinxparam{\DUrole{n}{event}}}{}
\pysigstopsignatures
\sphinxAtStartPar
Affiche l’annotation lorsque le texte est cliqué.

\end{fulllineitems}



\section{frise.save module}
\label{\detokenize{frise:module-frise.save}}\label{\detokenize{frise:frise-save-module}}\index{module@\spxentry{module}!frise.save@\spxentry{frise.save}}\index{frise.save@\spxentry{frise.save}!module@\spxentry{module}}
\sphinxAtStartPar
Ce fichier contient des classes et fonctions de base pour sauvegarder les fichiers ‘.txt’ et frise de sortie
\index{MetaData\_WD (class in frise.save)@\spxentry{MetaData\_WD}\spxextra{class in frise.save}}

\begin{fulllineitems}
\phantomsection\label{\detokenize{frise:frise.save.MetaData_WD}}
\pysigstartsignatures
\pysiglinewithargsret{\sphinxbfcode{\sphinxupquote{class\DUrole{w}{ }}}\sphinxcode{\sphinxupquote{frise.save.}}\sphinxbfcode{\sphinxupquote{MetaData\_WD}}}{\sphinxparam{\DUrole{n}{filename}}}{}
\pysigstopsignatures
\sphinxAtStartPar
Bases: \sphinxcode{\sphinxupquote{CTkToplevel}}

\sphinxAtStartPar
fenetre toplevel pour saisir les metadonnées
\index{liste (frise.save.MetaData\_WD attribute)@\spxentry{liste}\spxextra{frise.save.MetaData\_WD attribute}}

\begin{fulllineitems}
\phantomsection\label{\detokenize{frise:frise.save.MetaData_WD.liste}}
\pysigstartsignatures
\pysigline{\sphinxbfcode{\sphinxupquote{liste}}}
\pysigstopsignatures
\sphinxAtStartPar
liste de symptomes a sauvegarder
\begin{quote}\begin{description}
\sphinxlineitem{Type}
\sphinxAtStartPar
list

\end{description}\end{quote}

\end{fulllineitems}

\index{filename (frise.save.MetaData\_WD attribute)@\spxentry{filename}\spxextra{frise.save.MetaData\_WD attribute}}

\begin{fulllineitems}
\phantomsection\label{\detokenize{frise:frise.save.MetaData_WD.filename}}
\pysigstartsignatures
\pysigline{\sphinxbfcode{\sphinxupquote{filename}}}
\pysigstopsignatures
\sphinxAtStartPar
chemin du fichier dans lequel ecrire
\begin{quote}\begin{description}
\sphinxlineitem{Type}
\sphinxAtStartPar
string

\end{description}\end{quote}

\end{fulllineitems}

\index{get\_metadata() (frise.save.MetaData\_WD method)@\spxentry{get\_metadata()}\spxextra{frise.save.MetaData\_WD method}}

\begin{fulllineitems}
\phantomsection\label{\detokenize{frise:frise.save.MetaData_WD.get_metadata}}
\pysigstartsignatures
\pysiglinewithargsret{\sphinxbfcode{\sphinxupquote{get\_metadata}}}{\sphinxparam{\DUrole{n}{event}}}{}
\pysigstopsignatures
\sphinxAtStartPar
ecrire une liste des metadonnées sous la forme {[}heure réelle, patient, praticien{]}
\begin{quote}\begin{description}
\sphinxlineitem{Parameters}
\sphinxAtStartPar
\sphinxstyleliteralstrong{\sphinxupquote{event}} (\sphinxstyleliteralemphasis{\sphinxupquote{any}}) \textendash{} correspont a l’ecriture dans les box de texte

\end{description}\end{quote}

\end{fulllineitems}


\end{fulllineitems}

\index{save (class in frise.save)@\spxentry{save}\spxextra{class in frise.save}}

\begin{fulllineitems}
\phantomsection\label{\detokenize{frise:frise.save.save}}
\pysigstartsignatures
\pysiglinewithargsret{\sphinxbfcode{\sphinxupquote{class\DUrole{w}{ }}}\sphinxcode{\sphinxupquote{frise.save.}}\sphinxbfcode{\sphinxupquote{save}}}{\sphinxparam{\DUrole{n}{Liste\_symptomes}\DUrole{o}{=}\DUrole{default_value}{{[}{]}}}}{}
\pysigstopsignatures
\sphinxAtStartPar
Bases: \sphinxcode{\sphinxupquote{object}}

\sphinxAtStartPar
classe dédiée à la sauvegarde des fichiers
\index{symptomes (frise.save.save attribute)@\spxentry{symptomes}\spxextra{frise.save.save attribute}}

\begin{fulllineitems}
\phantomsection\label{\detokenize{frise:frise.save.save.symptomes}}
\pysigstartsignatures
\pysigline{\sphinxbfcode{\sphinxupquote{symptomes}}}
\pysigstopsignatures
\sphinxAtStartPar
liste des symptomes a sauvegarder
\begin{quote}\begin{description}
\sphinxlineitem{Type}
\sphinxAtStartPar
Liste

\end{description}\end{quote}

\end{fulllineitems}

\index{save() (frise.save.save method)@\spxentry{save()}\spxextra{frise.save.save method}}

\begin{fulllineitems}
\phantomsection\label{\detokenize{frise:frise.save.save.save}}
\pysigstartsignatures
\pysiglinewithargsret{\sphinxbfcode{\sphinxupquote{save}}}{}{}
\pysigstopsignatures
\sphinxAtStartPar
Enregistre les symptomes dans un fichier pour etre rechargés
\begin{quote}\begin{description}
\sphinxlineitem{Raises}
\sphinxAtStartPar
\sphinxstyleliteralstrong{\sphinxupquote{FileNotFoundError}} \textendash{} message d’erreur si echec de recuperation du chemin du fichier.

\end{description}\end{quote}

\end{fulllineitems}

\index{set\_symptomes() (frise.save.save method)@\spxentry{set\_symptomes()}\spxextra{frise.save.save method}}

\begin{fulllineitems}
\phantomsection\label{\detokenize{frise:frise.save.save.set_symptomes}}
\pysigstartsignatures
\pysiglinewithargsret{\sphinxbfcode{\sphinxupquote{set\_symptomes}}}{\sphinxparam{\DUrole{n}{Liste\_symptomes}}}{}
\pysigstopsignatures
\sphinxAtStartPar
Actualise les symptomes
\begin{quote}\begin{description}
\sphinxlineitem{Parameters}
\sphinxAtStartPar
\sphinxstyleliteralstrong{\sphinxupquote{Liste\_symptomes}} (\sphinxstyleliteralemphasis{\sphinxupquote{list}}) \textendash{} Liste de symptomes

\end{description}\end{quote}

\end{fulllineitems}

\index{write\_report() (frise.save.save method)@\spxentry{write\_report()}\spxextra{frise.save.save method}}

\begin{fulllineitems}
\phantomsection\label{\detokenize{frise:frise.save.save.write_report}}
\pysigstartsignatures
\pysiglinewithargsret{\sphinxbfcode{\sphinxupquote{write\_report}}}{}{}
\pysigstopsignatures
\sphinxAtStartPar
Ecrit un fichier lisible par un humain
\begin{quote}\begin{description}
\sphinxlineitem{Raises}
\sphinxAtStartPar
\sphinxstyleliteralstrong{\sphinxupquote{FileNotFoundError}} \textendash{} message d’erreur si echec de recuperation du chemin du fichier.

\end{description}\end{quote}

\end{fulllineitems}


\end{fulllineitems}



\section{Module contents}
\label{\detokenize{frise:module-frise}}\label{\detokenize{frise:module-contents}}\index{module@\spxentry{module}!frise@\spxentry{frise}}\index{frise@\spxentry{frise}!module@\spxentry{module}}
\sphinxstepscope


\chapter{annotation package}
\label{\detokenize{annotation:annotation-package}}\label{\detokenize{annotation::doc}}

\section{Submodules}
\label{\detokenize{annotation:submodules}}

\section{annotation.class\_Menu\_symptomes module}
\label{\detokenize{annotation:module-annotation.class_Menu_symptomes}}\label{\detokenize{annotation:annotation-class-menu-symptomes-module}}\index{module@\spxentry{module}!annotation.class\_Menu\_symptomes@\spxentry{annotation.class\_Menu\_symptomes}}\index{annotation.class\_Menu\_symptomes@\spxentry{annotation.class\_Menu\_symptomes}!module@\spxentry{module}}\index{MenuSymptomes (class in annotation.class\_Menu\_symptomes)@\spxentry{MenuSymptomes}\spxextra{class in annotation.class\_Menu\_symptomes}}

\begin{fulllineitems}
\phantomsection\label{\detokenize{annotation:annotation.class_Menu_symptomes.MenuSymptomes}}
\pysigstartsignatures
\pysiglinewithargsret{\sphinxbfcode{\sphinxupquote{class\DUrole{w}{ }}}\sphinxcode{\sphinxupquote{annotation.class\_Menu\_symptomes.}}\sphinxbfcode{\sphinxupquote{MenuSymptomes}}}{\sphinxparam{\DUrole{n}{master}}}{}
\pysigstopsignatures
\sphinxAtStartPar
Bases: \sphinxcode{\sphinxupquote{CTkFrame}}

\sphinxAtStartPar
Classe pour avoir les menus déroulants et la recherche textuelle
Les fichiers de symptomes doivent etre de la forme : Categorie; Symptome1; Symptome2; Symptome3
\index{filtrer\_options() (annotation.class\_Menu\_symptomes.MenuSymptomes method)@\spxentry{filtrer\_options()}\spxextra{annotation.class\_Menu\_symptomes.MenuSymptomes method}}

\begin{fulllineitems}
\phantomsection\label{\detokenize{annotation:annotation.class_Menu_symptomes.MenuSymptomes.filtrer_options}}
\pysigstartsignatures
\pysiglinewithargsret{\sphinxbfcode{\sphinxupquote{filtrer\_options}}}{\sphinxparam{\DUrole{n}{event}}}{}
\pysigstopsignatures
\sphinxAtStartPar
filtre les options d’un menu déroulant en fonction d’une recherche textuelle
affiche dans le menu déroulant la premiere option correspondante

\end{fulllineitems}


\end{fulllineitems}



\section{annotation.class\_symptome module}
\label{\detokenize{annotation:module-annotation.class_symptome}}\label{\detokenize{annotation:annotation-class-symptome-module}}\index{module@\spxentry{module}!annotation.class\_symptome@\spxentry{annotation.class\_symptome}}\index{annotation.class\_symptome@\spxentry{annotation.class\_symptome}!module@\spxentry{module}}\index{Symptome (class in annotation.class\_symptome)@\spxentry{Symptome}\spxextra{class in annotation.class\_symptome}}

\begin{fulllineitems}
\phantomsection\label{\detokenize{annotation:annotation.class_symptome.Symptome}}
\pysigstartsignatures
\pysiglinewithargsret{\sphinxbfcode{\sphinxupquote{class\DUrole{w}{ }}}\sphinxcode{\sphinxupquote{annotation.class\_symptome.}}\sphinxbfcode{\sphinxupquote{Symptome}}}{\sphinxparam{\DUrole{n}{ID}\DUrole{o}{=}\DUrole{default_value}{None}}\sphinxparamcomma \sphinxparam{\DUrole{n}{Name}\DUrole{o}{=}\DUrole{default_value}{None}}\sphinxparamcomma \sphinxparam{\DUrole{n}{Lateralization}\DUrole{o}{=}\DUrole{default_value}{None}}\sphinxparamcomma \sphinxparam{\DUrole{n}{Topography}\DUrole{o}{=}\DUrole{default_value}{None}}\sphinxparamcomma \sphinxparam{\DUrole{n}{Orientation}\DUrole{o}{=}\DUrole{default_value}{None}}\sphinxparamcomma \sphinxparam{\DUrole{n}{AttributSuppl}\DUrole{o}{=}\DUrole{default_value}{None}}\sphinxparamcomma \sphinxparam{\DUrole{n}{Tdeb}\DUrole{o}{=}\DUrole{default_value}{None}}\sphinxparamcomma \sphinxparam{\DUrole{n}{Tfin}\DUrole{o}{=}\DUrole{default_value}{None}}\sphinxparamcomma \sphinxparam{\DUrole{n}{Comment}\DUrole{o}{=}\DUrole{default_value}{None}}}{}
\pysigstopsignatures
\sphinxAtStartPar
Bases: \sphinxcode{\sphinxupquote{object}}

\sphinxAtStartPar
Classe permettant d’instancier un objet Symptome avec tous ses attributs
\index{ID (annotation.class\_symptome.Symptome attribute)@\spxentry{ID}\spxextra{annotation.class\_symptome.Symptome attribute}}

\begin{fulllineitems}
\phantomsection\label{\detokenize{annotation:annotation.class_symptome.Symptome.ID}}
\pysigstartsignatures
\pysigline{\sphinxbfcode{\sphinxupquote{ID}}}
\pysigstopsignatures\begin{quote}\begin{description}
\sphinxlineitem{Type}
\sphinxAtStartPar
str

\end{description}\end{quote}

\end{fulllineitems}

\index{Name (annotation.class\_symptome.Symptome attribute)@\spxentry{Name}\spxextra{annotation.class\_symptome.Symptome attribute}}

\begin{fulllineitems}
\phantomsection\label{\detokenize{annotation:annotation.class_symptome.Symptome.Name}}
\pysigstartsignatures
\pysigline{\sphinxbfcode{\sphinxupquote{Name}}}
\pysigstopsignatures\begin{quote}\begin{description}
\sphinxlineitem{Type}
\sphinxAtStartPar
str

\end{description}\end{quote}

\end{fulllineitems}

\index{Lateralization (annotation.class\_symptome.Symptome attribute)@\spxentry{Lateralization}\spxextra{annotation.class\_symptome.Symptome attribute}}

\begin{fulllineitems}
\phantomsection\label{\detokenize{annotation:annotation.class_symptome.Symptome.Lateralization}}
\pysigstartsignatures
\pysigline{\sphinxbfcode{\sphinxupquote{Lateralization}}}
\pysigstopsignatures\begin{quote}\begin{description}
\sphinxlineitem{Type}
\sphinxAtStartPar
str

\end{description}\end{quote}

\end{fulllineitems}



\begin{fulllineitems}

\pysigstartsignatures
\pysigline{\sphinxbfcode{\sphinxupquote{segment~corporel}}}
\pysigstopsignatures\begin{quote}\begin{description}
\sphinxlineitem{Type}
\sphinxAtStartPar
str

\end{description}\end{quote}

\end{fulllineitems}



\begin{fulllineitems}

\pysigstartsignatures
\pysigline{\sphinxbfcode{\sphinxupquote{Tps~debut}}}
\pysigstopsignatures\begin{quote}\begin{description}
\sphinxlineitem{Type}
\sphinxAtStartPar
str

\end{description}\end{quote}

\end{fulllineitems}



\begin{fulllineitems}

\pysigstartsignatures
\pysigline{\sphinxbfcode{\sphinxupquote{Tps~fin}}}
\pysigstopsignatures\begin{quote}\begin{description}
\sphinxlineitem{Type}
\sphinxAtStartPar
str

\end{description}\end{quote}

\end{fulllineitems}

\index{Orientation (annotation.class\_symptome.Symptome attribute)@\spxentry{Orientation}\spxextra{annotation.class\_symptome.Symptome attribute}}

\begin{fulllineitems}
\phantomsection\label{\detokenize{annotation:annotation.class_symptome.Symptome.Orientation}}
\pysigstartsignatures
\pysigline{\sphinxbfcode{\sphinxupquote{Orientation}}}
\pysigstopsignatures\begin{quote}\begin{description}
\sphinxlineitem{Type}
\sphinxAtStartPar
str

\end{description}\end{quote}

\end{fulllineitems}



\begin{fulllineitems}

\pysigstartsignatures
\pysigline{\sphinxbfcode{\sphinxupquote{Attributs~suppl}}}
\pysigstopsignatures\begin{quote}\begin{description}
\sphinxlineitem{Type}
\sphinxAtStartPar
str

\end{description}\end{quote}

\end{fulllineitems}

\index{Comment (annotation.class\_symptome.Symptome attribute)@\spxentry{Comment}\spxextra{annotation.class\_symptome.Symptome attribute}}

\begin{fulllineitems}
\phantomsection\label{\detokenize{annotation:annotation.class_symptome.Symptome.Comment}}
\pysigstartsignatures
\pysigline{\sphinxbfcode{\sphinxupquote{Comment}}}
\pysigstopsignatures\begin{quote}\begin{description}
\sphinxlineitem{Type}
\sphinxAtStartPar
str

\end{description}\end{quote}

\end{fulllineitems}

\index{get\_AttributSuppl() (annotation.class\_symptome.Symptome method)@\spxentry{get\_AttributSuppl()}\spxextra{annotation.class\_symptome.Symptome method}}

\begin{fulllineitems}
\phantomsection\label{\detokenize{annotation:annotation.class_symptome.Symptome.get_AttributSuppl}}
\pysigstartsignatures
\pysiglinewithargsret{\sphinxbfcode{\sphinxupquote{get\_AttributSuppl}}}{}{}
\pysigstopsignatures
\end{fulllineitems}

\index{get\_Comment() (annotation.class\_symptome.Symptome method)@\spxentry{get\_Comment()}\spxextra{annotation.class\_symptome.Symptome method}}

\begin{fulllineitems}
\phantomsection\label{\detokenize{annotation:annotation.class_symptome.Symptome.get_Comment}}
\pysigstartsignatures
\pysiglinewithargsret{\sphinxbfcode{\sphinxupquote{get\_Comment}}}{}{}
\pysigstopsignatures
\end{fulllineitems}

\index{get\_ID() (annotation.class\_symptome.Symptome method)@\spxentry{get\_ID()}\spxextra{annotation.class\_symptome.Symptome method}}

\begin{fulllineitems}
\phantomsection\label{\detokenize{annotation:annotation.class_symptome.Symptome.get_ID}}
\pysigstartsignatures
\pysiglinewithargsret{\sphinxbfcode{\sphinxupquote{get\_ID}}}{}{}
\pysigstopsignatures
\end{fulllineitems}

\index{get\_Lateralization() (annotation.class\_symptome.Symptome method)@\spxentry{get\_Lateralization()}\spxextra{annotation.class\_symptome.Symptome method}}

\begin{fulllineitems}
\phantomsection\label{\detokenize{annotation:annotation.class_symptome.Symptome.get_Lateralization}}
\pysigstartsignatures
\pysiglinewithargsret{\sphinxbfcode{\sphinxupquote{get\_Lateralization}}}{}{}
\pysigstopsignatures
\end{fulllineitems}

\index{get\_Name() (annotation.class\_symptome.Symptome method)@\spxentry{get\_Name()}\spxextra{annotation.class\_symptome.Symptome method}}

\begin{fulllineitems}
\phantomsection\label{\detokenize{annotation:annotation.class_symptome.Symptome.get_Name}}
\pysigstartsignatures
\pysiglinewithargsret{\sphinxbfcode{\sphinxupquote{get\_Name}}}{}{}
\pysigstopsignatures
\end{fulllineitems}

\index{get\_Orientation() (annotation.class\_symptome.Symptome method)@\spxentry{get\_Orientation()}\spxextra{annotation.class\_symptome.Symptome method}}

\begin{fulllineitems}
\phantomsection\label{\detokenize{annotation:annotation.class_symptome.Symptome.get_Orientation}}
\pysigstartsignatures
\pysiglinewithargsret{\sphinxbfcode{\sphinxupquote{get\_Orientation}}}{}{}
\pysigstopsignatures
\end{fulllineitems}

\index{get\_Tdeb() (annotation.class\_symptome.Symptome method)@\spxentry{get\_Tdeb()}\spxextra{annotation.class\_symptome.Symptome method}}

\begin{fulllineitems}
\phantomsection\label{\detokenize{annotation:annotation.class_symptome.Symptome.get_Tdeb}}
\pysigstartsignatures
\pysiglinewithargsret{\sphinxbfcode{\sphinxupquote{get\_Tdeb}}}{}{}
\pysigstopsignatures
\end{fulllineitems}

\index{get\_Tfin() (annotation.class\_symptome.Symptome method)@\spxentry{get\_Tfin()}\spxextra{annotation.class\_symptome.Symptome method}}

\begin{fulllineitems}
\phantomsection\label{\detokenize{annotation:annotation.class_symptome.Symptome.get_Tfin}}
\pysigstartsignatures
\pysiglinewithargsret{\sphinxbfcode{\sphinxupquote{get\_Tfin}}}{}{}
\pysigstopsignatures
\end{fulllineitems}

\index{get\_Topography() (annotation.class\_symptome.Symptome method)@\spxentry{get\_Topography()}\spxextra{annotation.class\_symptome.Symptome method}}

\begin{fulllineitems}
\phantomsection\label{\detokenize{annotation:annotation.class_symptome.Symptome.get_Topography}}
\pysigstartsignatures
\pysiglinewithargsret{\sphinxbfcode{\sphinxupquote{get\_Topography}}}{}{}
\pysigstopsignatures
\end{fulllineitems}

\index{get\_attributs() (annotation.class\_symptome.Symptome method)@\spxentry{get\_attributs()}\spxextra{annotation.class\_symptome.Symptome method}}

\begin{fulllineitems}
\phantomsection\label{\detokenize{annotation:annotation.class_symptome.Symptome.get_attributs}}
\pysigstartsignatures
\pysiglinewithargsret{\sphinxbfcode{\sphinxupquote{get\_attributs}}}{}{}
\pysigstopsignatures
\sphinxAtStartPar
Retourne une liste contenant les valeurs des attributs de l’objet
\begin{quote}\begin{description}
\sphinxlineitem{Returns}
\sphinxAtStartPar

\sphinxAtStartPar
{[}
ID,
Name,
Lateralization,
Topography,
Orientation,
AttributSuppl,
Tdeb,
Tfin,
Comment
{]}


\sphinxlineitem{Return type}
\sphinxAtStartPar
\sphinxcode{\sphinxupquote{list}} of \sphinxcode{\sphinxupquote{str}}

\end{description}\end{quote}

\end{fulllineitems}

\index{set\_AttributSuppl() (annotation.class\_symptome.Symptome method)@\spxentry{set\_AttributSuppl()}\spxextra{annotation.class\_symptome.Symptome method}}

\begin{fulllineitems}
\phantomsection\label{\detokenize{annotation:annotation.class_symptome.Symptome.set_AttributSuppl}}
\pysigstartsignatures
\pysiglinewithargsret{\sphinxbfcode{\sphinxupquote{set\_AttributSuppl}}}{\sphinxparam{\DUrole{n}{new\_AttributSuppl}}}{}
\pysigstopsignatures
\end{fulllineitems}

\index{set\_Comment() (annotation.class\_symptome.Symptome method)@\spxentry{set\_Comment()}\spxextra{annotation.class\_symptome.Symptome method}}

\begin{fulllineitems}
\phantomsection\label{\detokenize{annotation:annotation.class_symptome.Symptome.set_Comment}}
\pysigstartsignatures
\pysiglinewithargsret{\sphinxbfcode{\sphinxupquote{set\_Comment}}}{\sphinxparam{\DUrole{n}{new\_Comment}}}{}
\pysigstopsignatures
\end{fulllineitems}

\index{set\_ID() (annotation.class\_symptome.Symptome method)@\spxentry{set\_ID()}\spxextra{annotation.class\_symptome.Symptome method}}

\begin{fulllineitems}
\phantomsection\label{\detokenize{annotation:annotation.class_symptome.Symptome.set_ID}}
\pysigstartsignatures
\pysiglinewithargsret{\sphinxbfcode{\sphinxupquote{set\_ID}}}{\sphinxparam{\DUrole{n}{new\_ID}}}{}
\pysigstopsignatures
\end{fulllineitems}

\index{set\_Lateralization() (annotation.class\_symptome.Symptome method)@\spxentry{set\_Lateralization()}\spxextra{annotation.class\_symptome.Symptome method}}

\begin{fulllineitems}
\phantomsection\label{\detokenize{annotation:annotation.class_symptome.Symptome.set_Lateralization}}
\pysigstartsignatures
\pysiglinewithargsret{\sphinxbfcode{\sphinxupquote{set\_Lateralization}}}{\sphinxparam{\DUrole{n}{new\_Lateralization}}}{}
\pysigstopsignatures
\end{fulllineitems}

\index{set\_Name() (annotation.class\_symptome.Symptome method)@\spxentry{set\_Name()}\spxextra{annotation.class\_symptome.Symptome method}}

\begin{fulllineitems}
\phantomsection\label{\detokenize{annotation:annotation.class_symptome.Symptome.set_Name}}
\pysigstartsignatures
\pysiglinewithargsret{\sphinxbfcode{\sphinxupquote{set\_Name}}}{\sphinxparam{\DUrole{n}{new\_Name}}}{}
\pysigstopsignatures
\end{fulllineitems}

\index{set\_Orientation() (annotation.class\_symptome.Symptome method)@\spxentry{set\_Orientation()}\spxextra{annotation.class\_symptome.Symptome method}}

\begin{fulllineitems}
\phantomsection\label{\detokenize{annotation:annotation.class_symptome.Symptome.set_Orientation}}
\pysigstartsignatures
\pysiglinewithargsret{\sphinxbfcode{\sphinxupquote{set\_Orientation}}}{\sphinxparam{\DUrole{n}{new\_Orientation}}}{}
\pysigstopsignatures
\end{fulllineitems}

\index{set\_Tdeb() (annotation.class\_symptome.Symptome method)@\spxentry{set\_Tdeb()}\spxextra{annotation.class\_symptome.Symptome method}}

\begin{fulllineitems}
\phantomsection\label{\detokenize{annotation:annotation.class_symptome.Symptome.set_Tdeb}}
\pysigstartsignatures
\pysiglinewithargsret{\sphinxbfcode{\sphinxupquote{set\_Tdeb}}}{\sphinxparam{\DUrole{n}{new\_Tdeb}}}{}
\pysigstopsignatures
\end{fulllineitems}

\index{set\_Tfin() (annotation.class\_symptome.Symptome method)@\spxentry{set\_Tfin()}\spxextra{annotation.class\_symptome.Symptome method}}

\begin{fulllineitems}
\phantomsection\label{\detokenize{annotation:annotation.class_symptome.Symptome.set_Tfin}}
\pysigstartsignatures
\pysiglinewithargsret{\sphinxbfcode{\sphinxupquote{set\_Tfin}}}{\sphinxparam{\DUrole{n}{new\_Tfin}}}{}
\pysigstopsignatures
\end{fulllineitems}

\index{set\_Topography() (annotation.class\_symptome.Symptome method)@\spxentry{set\_Topography()}\spxextra{annotation.class\_symptome.Symptome method}}

\begin{fulllineitems}
\phantomsection\label{\detokenize{annotation:annotation.class_symptome.Symptome.set_Topography}}
\pysigstartsignatures
\pysiglinewithargsret{\sphinxbfcode{\sphinxupquote{set\_Topography}}}{\sphinxparam{\DUrole{n}{new\_Topography}}}{}
\pysigstopsignatures
\end{fulllineitems}


\end{fulllineitems}



\section{annotation.menu\_deroulant\_anaelle module}
\label{\detokenize{annotation:annotation-menu-deroulant-anaelle-module}}

\section{annotation.pop\_up module}
\label{\detokenize{annotation:module-annotation.pop_up}}\label{\detokenize{annotation:annotation-pop-up-module}}\index{module@\spxentry{module}!annotation.pop\_up@\spxentry{annotation.pop\_up}}\index{annotation.pop\_up@\spxentry{annotation.pop\_up}!module@\spxentry{module}}\index{SymptomeEditor (class in annotation.pop\_up)@\spxentry{SymptomeEditor}\spxextra{class in annotation.pop\_up}}

\begin{fulllineitems}
\phantomsection\label{\detokenize{annotation:annotation.pop_up.SymptomeEditor}}
\pysigstartsignatures
\pysiglinewithargsret{\sphinxbfcode{\sphinxupquote{class\DUrole{w}{ }}}\sphinxcode{\sphinxupquote{annotation.pop\_up.}}\sphinxbfcode{\sphinxupquote{SymptomeEditor}}}{\sphinxparam{\DUrole{n}{Symp}}}{}
\pysigstopsignatures
\sphinxAtStartPar
Bases: \sphinxcode{\sphinxupquote{CTkToplevel}}

\sphinxAtStartPar
Permet l’edition de symptomes au travers d’une fenetre
\index{apply\_changes() (annotation.pop\_up.SymptomeEditor method)@\spxentry{apply\_changes()}\spxextra{annotation.pop\_up.SymptomeEditor method}}

\begin{fulllineitems}
\phantomsection\label{\detokenize{annotation:annotation.pop_up.SymptomeEditor.apply_changes}}
\pysigstartsignatures
\pysiglinewithargsret{\sphinxbfcode{\sphinxupquote{apply\_changes}}}{\sphinxparam{\DUrole{n}{Symp}}}{}
\pysigstopsignatures
\sphinxAtStartPar
Récupère les valeurs saisies et met à jour les attributs de la classe Symp

\end{fulllineitems}

\index{create\_entry() (annotation.pop\_up.SymptomeEditor method)@\spxentry{create\_entry()}\spxextra{annotation.pop\_up.SymptomeEditor method}}

\begin{fulllineitems}
\phantomsection\label{\detokenize{annotation:annotation.pop_up.SymptomeEditor.create_entry}}
\pysigstartsignatures
\pysiglinewithargsret{\sphinxbfcode{\sphinxupquote{create\_entry}}}{\sphinxparam{\DUrole{n}{label\_text}}\sphinxparamcomma \sphinxparam{\DUrole{n}{initial\_value}}\sphinxparamcomma \sphinxparam{\DUrole{n}{i}}}{}
\pysigstopsignatures
\sphinxAtStartPar
crée et place des zones de texte (entrées) sur la fenêtre
\begin{quote}\begin{description}
\sphinxlineitem{Parameters}\begin{itemize}
\item {} 
\sphinxAtStartPar
\sphinxstyleliteralstrong{\sphinxupquote{label\_text}} (\sphinxstyleliteralemphasis{\sphinxupquote{str}}) \textendash{} label de l’entrée

\item {} 
\sphinxAtStartPar
\sphinxstyleliteralstrong{\sphinxupquote{initial\_value}} (\sphinxstyleliteralemphasis{\sphinxupquote{str}}) \textendash{} valeur actuelle de l’entrée

\item {} 
\sphinxAtStartPar
\sphinxstyleliteralstrong{\sphinxupquote{i}} (\sphinxstyleliteralemphasis{\sphinxupquote{int}}) \textendash{} indice de la ligne de l’entrée

\end{itemize}

\end{description}\end{quote}

\end{fulllineitems}


\end{fulllineitems}



\section{annotation.recherche\_menu\_déroulant module}
\label{\detokenize{annotation:module-annotation.recherche_menu_deroulant}}\label{\detokenize{annotation:annotation-recherche-menu-deroulant-module}}\index{module@\spxentry{module}!annotation.recherche\_menu\_déroulant@\spxentry{annotation.recherche\_menu\_déroulant}}\index{annotation.recherche\_menu\_déroulant@\spxentry{annotation.recherche\_menu\_déroulant}!module@\spxentry{module}}\index{filtrer\_options() (in module annotation.recherche\_menu\_déroulant)@\spxentry{filtrer\_options()}\spxextra{in module annotation.recherche\_menu\_déroulant}}

\begin{fulllineitems}
\phantomsection\label{\detokenize{annotation:annotation.recherche_menu_deroulant.filtrer_options}}
\pysigstartsignatures
\pysiglinewithargsret{\sphinxcode{\sphinxupquote{annotation.recherche\_menu\_déroulant.}}\sphinxbfcode{\sphinxupquote{filtrer\_options}}}{\sphinxparam{\DUrole{n}{event}}}{}
\pysigstopsignatures
\sphinxAtStartPar
filtre les options d’un menu déroulant en fonction d’une recherche textuelle

\end{fulllineitems}



\section{annotation.symptom\_label module}
\label{\detokenize{annotation:annotation-symptom-label-module}}

\section{Module contents}
\label{\detokenize{annotation:module-annotation}}\label{\detokenize{annotation:module-contents}}\index{module@\spxentry{module}!annotation@\spxentry{annotation}}\index{annotation@\spxentry{annotation}!module@\spxentry{module}}
\sphinxstepscope


\chapter{lecteur\_video package}
\label{\detokenize{lecteur_video:lecteur-video-package}}\label{\detokenize{lecteur_video::doc}}

\section{Submodules}
\label{\detokenize{lecteur_video:submodules}}

\section{lecteur\_video.barre\_progression module}
\label{\detokenize{lecteur_video:lecteur-video-barre-progression-module}}

\section{lecteur\_video.csacade module}
\label{\detokenize{lecteur_video:lecteur-video-csacade-module}}

\section{lecteur\_video.interface\_menu\_deroulants module}
\label{\detokenize{lecteur_video:lecteur-video-interface-menu-deroulants-module}}

\section{lecteur\_video.menu\_indente module}
\label{\detokenize{lecteur_video:lecteur-video-menu-indente-module}}

\section{Module contents}
\label{\detokenize{lecteur_video:module-lecteur_video}}\label{\detokenize{lecteur_video:module-contents}}\index{module@\spxentry{module}!lecteur\_video@\spxentry{lecteur\_video}}\index{lecteur\_video@\spxentry{lecteur\_video}!module@\spxentry{module}}

\chapter{Indices and tables}
\label{\detokenize{index:indices-and-tables}}\begin{itemize}
\item {} 
\sphinxAtStartPar
\DUrole{xref,std,std-ref}{genindex}

\item {} 
\sphinxAtStartPar
\DUrole{xref,std,std-ref}{modindex}

\item {} 
\sphinxAtStartPar
\DUrole{xref,std,std-ref}{search}

\end{itemize}


\renewcommand{\indexname}{Python Module Index}
\begin{sphinxtheindex}
\let\bigletter\sphinxstyleindexlettergroup
\bigletter{a}
\item\relax\sphinxstyleindexentry{annotation}\sphinxstyleindexpageref{annotation:\detokenize{module-annotation}}
\item\relax\sphinxstyleindexentry{annotation.class\_Menu\_symptomes}\sphinxstyleindexpageref{annotation:\detokenize{module-annotation.class_Menu_symptomes}}
\item\relax\sphinxstyleindexentry{annotation.class\_symptome}\sphinxstyleindexpageref{annotation:\detokenize{module-annotation.class_symptome}}
\item\relax\sphinxstyleindexentry{annotation.pop\_up}\sphinxstyleindexpageref{annotation:\detokenize{module-annotation.pop_up}}
\item\relax\sphinxstyleindexentry{annotation.recherche\_menu\_déroulant}\sphinxstyleindexpageref{annotation:\detokenize{module-annotation.recherche_menu_deroulant}}
\indexspace
\bigletter{f}
\item\relax\sphinxstyleindexentry{frise}\sphinxstyleindexpageref{frise:\detokenize{module-frise}}
\item\relax\sphinxstyleindexentry{frise.ecriture\_fichier}\sphinxstyleindexpageref{frise:\detokenize{module-frise.ecriture_fichier}}
\item\relax\sphinxstyleindexentry{frise.fonctions\_frise}\sphinxstyleindexpageref{frise:\detokenize{module-frise.fonctions_frise}}
\item\relax\sphinxstyleindexentry{frise.save}\sphinxstyleindexpageref{frise:\detokenize{module-frise.save}}
\indexspace
\bigletter{g}
\item\relax\sphinxstyleindexentry{general\_interface\_V9}\sphinxstyleindexpageref{general_interface:\detokenize{module-general_interface_V9}}
\indexspace
\bigletter{l}
\item\relax\sphinxstyleindexentry{lecteur\_video}\sphinxstyleindexpageref{lecteur_video:\detokenize{module-lecteur_video}}
\end{sphinxtheindex}

\renewcommand{\indexname}{Index}
\printindex
\end{document}